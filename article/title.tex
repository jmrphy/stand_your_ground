\documentclass[12pt, oneside]{article}   	% use "amsart" instead of "article" for AMSLaTeX format
\usepackage{geometry}                		% See geometry.pdf to learn the layout options. There are lots.
\geometry{letterpaper}                   		% ... or a4paper or a5paper or ... 
%\geometry{landscape}                		% Activate for for rotated page geometry
%\usepackage[parfill]{parskip}    		% Activate to begin paragraphs with an empty line rather than an indent
\usepackage{graphicx}				% Use pdf, png, jpg, or eps§ with pdflatex; use eps in DVI mode
\usepackage{hyperref}
\hypersetup{colorlinks=false, linkcolor=blue}		
\usepackage{amssymb}

\title{Are ``Stand Your Ground'' Laws Racist and Sexist? A Statistical Analysis of Cases in Florida, 2005-2013\footnote{Direct all correspondence to Justin Murphy, University of Southampton, Politics and International Relations, Building 58, Room 3083, University Road, Southampton SO17 1BJ, United Kingdom. Email: \href{mailto:j.murphy@soton.ac.uk}{j.murphy@soton.ac.uk. I am grateful to Kevin Arceneaux for reading and providing useful feedback on earlier drafts of this manuscript. All of the
  data and code required to reproduce this article are available in the
  code repository at
  \href{http://j.mp/syg_repository}{j.mp/syg\_repository}. In
  particular, all of the code for processing the raw \emph{Tampa Bay
  Times} data I scraped from the web is at
  \href{http://j.mp/clean_syg}{j.mp/clean\_syg}; the full, final
  spreadsheet of all variables for all 237 cases made available by the
  \emph{Tampa Bay Times} is at
  \href{http://j.mp/syg_data}{j.mp/syg\_data}; the subset of cases used
  in the analyses below can be found at
  \href{http://j.mp/model_data}{j.mp/model\_data}.
}}}

\author{Justin Murphy\\University of Southampton}
\date{}

\begin{document}
\maketitle

\section*{Abstract}
\textbf{Objective}: I test for racial and gender bias in the enforcement of ``stand your ground" (SYG) laws, controlling for potential confounders often invoked to reject claims of racism and sexism. \textbf{Method}: Regressions, simulations, and genetic matching are conducted using case-level data from 237 incidents in the US state of Florida between 2005 and 2013. \textbf{Results}: Controlling for potential confounders, the probability of conviction for a white defendant against a white victim is an estimated 90\% with much error; for a black defendant it is nearly 100\% with little error. For a male defendant in a domestic case, the probability is 40\% whereas for a female defendant it is 80\%. \textbf{Conclusions}: Enforcement of SYG laws appears biased against people of color in general and women specifically in the home. Policy implications are especially stark because these findings contradict recent research conducted for the US Senate. 




\end{document}  