\begin{abstract}
In the United States, ``stand your ground" (SYG) laws have been adopted by most states with the supposed intention of empowering self-defense, yet critics argue these laws reinforce white supremacy in the public sphere and male supremacy in the home. This research note presents the first statistical tests for racial and gender bias in the enforcement of SYG laws to control for a wide variety of other factors frequently cited to justify observed outcomes. Considering a sample of SYG cases in Florida from 2006-2013, I find the probability of conviction for a white defendant against a white victim in a typical case is fairly high at around 90\% but with a large margin of error, whereas the probability of conviction for a black defendant against a white victim approaches 100\%, even after controlling for more than 10 different objective factors related to the circumstances of the incident. The probability of conviction for a male defendant in a typical domestic case is found to be about 40\%, but for a female defendant in an otherwise objectively equivalent case the probability of conviction increases dramatically to 80\%. I estimate that the probability George Zimmerman was going to be found guilty of murdering unarmed black teenager Trayvon Martin in 2012 was marginally greater than 50\% but would have been about 98\% if Trayvon Martin had been white. On the other hand, black female Marissa Alexander who fired a physically innocuous warning shot to deter her husband in 2010 faced a probability of conviction marginally greater than 50\%, but the probability of conviction for a male defendant in an otherwise objectively equivalent situation would have been around only 12\%. Finally, I show that these results are not due to outliers or non-random assignment. This research has important implications for scholars, lawmakers, judges, and activists because it provides new and improved evidence that SYG laws contribute to the legal institutionalization of racism and sexism.\footnote{Justin Murphy (\url{jmrphy.net}, \href{http://twitter.com/jmrphy}{@jmrphy}) is Assistant Professor of Politics at University of Southampton, UK. This is the first release of a working paper and has not been peer-reviewed; please send questions and comments to \href{mailto:j.murphy@soton.ac.uk}{j.murphy@soton.ac.uk}.}
\end{abstract}
\onehalfspacing