\begin{abstract}
Objective: I test for racial and gender bias in the enforcement of ``stand your ground" (SYG) laws, controlling for potential confounders often invoked to reject claims of racism and sexism. Method: Regressions, simulations, and genetic matching are conducted using case-level data from 237 incidents in the US state of Florida between 2006 and 2013. Results: Controlling for potential confounders, the probability of conviction for a white defendant against a white victim is an estimated 90\% with much error; for a black defendant it is nearly 100\% with little error. For a male defendant in a domestic case, the probability is 40\% whereas for a female defendant it is 80\%. Conclusions: Enforcement of SYG laws appears biased against people of color in general and women specifically in the home. Policy implications are especially stark because these findings contradict recent research conducted for the US Senate.\end{abstract}
\doublespacing