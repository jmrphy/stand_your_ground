\begin{abstract}
In the United States, ``stand your ground" laws have been adopted by most states with the stated intention of empowering self-defense, yet an increasing chorus of critics argue that they effectively enforce white supremacy. Surprisingly, the only previous statistical research to consider the issue at the level of individual cases is a short analysis submitted in testimony to the US Senate by conservative gun advocate John Lott. Though Lott finds no evidence of racial bias, I show that John Lott's study was fundamentally flawed and, using the same data, I find very robust evidence of racial bias in ``stand your ground" cases in Florida from 2006-2013. In particular, I find that cases with a white victim are far more likely to end with conviction than cases in which the victim is a person of color, even after accounting for up to 16 other factors including weaponry, whether the vicitim initiated, whether the victim died, etc.. Further, when one considers the race of the defendant, the bias toward conviction in cases of white victims is significantly greater for defendants of color than for white defendants. Finally, while in general a key predictor of conviction is whether the victim initiated the altercation, the fact of victim-initiation is more likely to lead to conviction when the victim is a person of color.\footnote{Justin Murphy (\url{jmrphy.net}, \href{http://twitter.com/jmrphy}{@jmrphy}) is Assistant Professor of Politics at University of Southampton, UK. Questions and comments may be sent to \href{mailto:j.murphy@soton.ac.uk}{j.murphy@soton.ac.uk}.}
\end{abstract}
\onehalfspacing