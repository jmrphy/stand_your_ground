\begin{abstract}
In the U.S., ``stand your ground" laws have been adopted by most states with the supposed intention of empowering self-defense, yet critics argue these laws re-enforce white supremacy in the public sphere and male supremacy in the home. This research note presents the first statistical test for racial and gender bias in the enforcement of ``stand your ground" laws which also controls for a wide variety of other factors frequently cited to justify observed outcomes. I find evidence of racial and gender bias in ``stand your ground" cases in Florida from 2006-2013. The probability of conviction for a white defendant against a white victim is found to be fairly high at around 90\% but with a large margin of error while the probability of conviction for a black defendant is nearly 100\% with a very small margin of error. The probability of conviction for a male defendant in a typical domestic case was found to be about 40\%, but for a female defendant in an otherwise objectively equivalent case the probability of conviction increases dramatically to 80\%. This research has important implications for legal scholars, lawmakers, judges, and activists because it provides new evidence that in practice ``stand your ground" laws may contribute to the legal institutionalization of racism and sexism. In particular, statistical simulations suggest the probability George Zimmerman was going to be found guilty of murdering unarmed black teenager Trayvon Martin in 2012 was marginally greater than 50\% but would have been about 98\% if Trayvon Martin had been white. On the other hand, black female Marissa Alexander who fired a physically innocuous warning shot to deter her husband in 2010 faced a probability of conviction marginally greater than 50\%, but the probability of conviction for a male defendant in an otherwise objectively equivalent situation would have been a mere 12\%. Thus the findings are especially important for currently-open and future cases such as Marissa Alexander's in which sexism or racism could independently lead to incorrect convictions. \footnote{Justin Murphy (\url{jmrphy.net}, \href{http://twitter.com/jmrphy}{@jmrphy}) is Assistant Professor of Politics at University of Southampton, UK. This is the first release of a working paper and has not been peer-reviewed; please send questions and comments to \href{mailto:j.murphy@soton.ac.uk}{j.murphy@soton.ac.uk}.}
\end{abstract}
\onehalfspacing